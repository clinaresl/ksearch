\begin{tabular}{c|cccccccccccc}\toprule
\multicolumn{13}{c}{#Centroids - Maps 25 unit}\\ \midrule
Algorithm & k=1 & k=2 & k=3 & k=4 & k=5 & k=10 & k=50 & k=100 & k=500 & k=1000 & k=5000 & k=10000 \\ \midrule
BELA$^*$ & \textbf{1.00} & \textbf{1.68} & \textbf{2.37} & \textbf{2.87} & \textbf{3.38} & \textbf{4.83} & \textbf{9.09} & \textbf{10.76} & \textbf{15.51} & \textbf{18.40} & \textbf{23.41} & \textbf{25.58} \\
K$^*$ & -- & -- & -- & -- & -- & -- & -- & -- & -- & -- & -- & -- \\
mA$^*$ & -- & -- & -- & -- & -- & -- & -- & -- & -- & -- & -- & -- \\ \bottomrule 
\end{tabular}
